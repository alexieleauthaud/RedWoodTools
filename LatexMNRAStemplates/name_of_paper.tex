\documentclass[useAMS,usenatbib,twocolumn]{mnras}
\usepackage{natbib}
\citestyle{aa}
\usepackage{graphicx}
\usepackage{times}
\usepackage{rotate}
\usepackage{hyperref}
%\usepackage{breakurl}
\usepackage{url} 
%\voffset=-0.4in

\newcommand{\cgs}{ ${\rm erg~cm}^{-2}~{\rm s}^{-1}$} 
\newcommand{\lum}{\rm erg~s$^{-1}$}
\def\xmm{{XMM-{\it Newton}}}
\def\chandra{{\it Chandra}}

\newcommand{\redm}{redMaPPer }
\newcommand{\myemail}{alexie.leauthaud@ipmu.jp}
\newcommand{\flux}{erg cm$^{-2}$ s$^{-1}$ }
\newcommand{\mass}{ M_{\sun}}

\newcommand{\etal}{et al.}
\newcommand{\eg}{{\it e.g.,}}
\newcommand{\cf}{{\it c.f.}}
\newcommand{\ie}{{\it i.e.}}
\newcommand{\etc}{{\it etc.}}
\newcommand{\sn}{S{\rm /}N}
\newcommand{\rmd}{{\rm d}}
\newcommand{\Msol}{\mbox{$M_{\odot}$}}
\newcommand{\msun}{\mbox{$M_{\odot}$}}
\newcommand{\Msun}{\mbox{${\bf M_{\odot} }$}}
\newcommand{\Lsol}{\mbox{$L_{\odot}$}}
\newcommand{\lsun}{\mbox{$L_{\odot}$}}
\newcommand{\ergsec}{\mbox{erg s$^{-1}$}}
\newcommand{\sm}{$~{\rm M}_{\odot}$~}
\newcommand{\hmsol}{$h~{\rm M}_{\odot}$~}

\newcommand{\cm}{\ensuremath{\mathrm{cm}}}
\newcommand{\km}{\ensuremath{\mathrm{km}}}
\newcommand{\kms}{\ensuremath{\mathrm{km\,s}^{-1}}}

\newcommand{\pc}{\ensuremath{\mathrm{pc}}}
\newcommand{\kpc}{\ensuremath{\mathrm{kpc}}}
\newcommand{\Mpc}{\ensuremath{\mathrm{Mpc}}}
\newcommand{\Gpc}{\ensuremath{\mathrm{Gpc}}}
\newcommand{\Angs}{\ensuremath{\text{\AA}}}

\newcommand{\Msolar}{\ensuremath{\mathrm{M}_\odot}}
\newcommand{\Lsolar}{\ensuremath{\mathrm{L}_\odot}}

\newcommand{\ds}{\ensuremath{\Delta\Sigma}}
\newcommand{\photoz}{photo-$z$~}
\newcommand{\photozs}{photo-$z$s~}
\newcommand{\zphot}{\ensuremath{z_\mathrm{phot}}}
\newcommand{\dsrand}{\ensuremath{\Delta\Sigma_\mathrm{rand}}}
\newcommand{\dsx}{\ensuremath{\Delta\Sigma_\times}}
\newcommand{\h}{\ensuremath{h^{-1}}~}
\def\lcdm{\Lambda{\rm CDM}}

\def\C#1{{\bf #1}}

\newcommand{\textcol}{\textcolor{black}{}}

% For US letter size
%\special{papersize=8.5in,11in}
%\setlength{\pdfpageheight}{\paperheight}
%\setlength{\pdfpagewidth}{\paperwidth}
%\setlength{\topmargin}{0.5in} 
 
%-------- BEGIN DOC  ---------------------
 
 \begin{document}
  
%-------- TITLE  ---------------------


\title[nice title]{Nice Title}



 %% LaTeX will automatically break titles if they run longer than
 %% one line. However, you may use \\ to force a line break if
 %% you desire.

%-------- AUTHORS  ---------------------

 %% Use \author, \affil, and the \and command to format
 %% author and affiliation information.
 %% Note that \email has replaced the old \authoremail command
 %% from AASTeX v4.0. You can use \email to mark an email address
 %% anywhere in the paper, not just in the front matter.
 %% As in the title, use \\ to force line breaks.

%% you might have to use the \newauthor command to break the lines of author names
 \author[someone et al.]  {your name$^{1}$\newauthor
\\
$^1$Department of Astronomy and Astrophysics, University of California, Santa Cruz, 1156 High Street, Santa Cruz, CA 95064 USA}


\maketitle
\label{firstpage}

%-------- ABSTRACT  ---------------------
 
\begin{abstract} This paper is awesome \end{abstract}

 
\begin{keywords}
cosmology: observations -- gravitational lensing -- large-scale structure of Universe
\end{keywords}
 
%-------- KEY WORDS  ---------------------

 %% Keywords should appear after the \end{abstract} command. The uncommented
 %% example has been keyed in ApJ style. See the instructions to authors
 %% for the journal to which you are submitting your paper to determine
 %% what keyword punctuation is appropriate.
 
%\keywords{cosmology: observations -- gravitational lensing -- large-scale
%structure of Universe}
 
 %% Authors who wish to have the most important objects in their paper
 %% linked in the electronic edition to a data center may do so by tagging
 %% their objects with \objectname{} or \object{}.  Each macro takes the
 %% object name as its required argument. The optional, square-bracket 
 %% argument should be used in cases where the data center identification
 %% differs from what is to be printed in the paper.  The text appearing 
 %% in curly braces is what will appear in print in the published paper. 
 %% If the object name is recognized by the data centers, it will be linked
 %% in the electronic edition to the object data available at the data centers

%%%%%%%%%%%%%%%%%%%%%%%%%%%%%%%%%%%%%%%%%%%%%%%%%%%%%%%%%%%%%%%%%%%%%%%%%%%%%%
%     INTRODUCTION
%%%%%%%%%%%%%%%%%%%%%%%%%%%%%%%%%%%%%%%%%%%%%%%%%%%%%%%%%%%%%%%%%%%%%%%%%%%%%%

%\clearpage

\section{Introduction}



%%%%%%%%%%%%%%%%%%%%%%%%%%%%%%%%%%%%%%%%%%%%%%%%%%%%%%%%%%%%%%%%%%%%%%%%%%%%%%
%     A Section
%%%%%%%%%%%%%%%%%%%%%%%%%%%%%%%%%%%%%%%%%%%%%%%%%%%%%%%%%%%%%%%%%%%%%%%%%%%%%%
\section{A section}\label{thissection}

% Example of how to include a figure 
%\begin{figure*}
%\begin{center}
%\includegraphics[width=18cm]{bpz_zphot_zspec.eps}
%\caption{a caption}
%\label{zphotzspec}
%\end{center}
%\end{figure*}


% Example of a Table
%\begin{table*}
%  \caption{a caption}
%\begin{tabular}{@{}cccccc}
%\hline
%Bin number &R [h$^{-1}$ Mpc] & $\Delta\Sigma$ [h M$_{\odot}$
%pc$^{-2}$]&$\Delta\Sigma$  [h M$_{\odot}$ pc$^{-2}$] & $\Delta\Sigma$  [h M$_{\odot}$ pc$^{-2}$]&
%$\Delta\Sigma$  [h M$_{\odot}$ pc$^{-2}$]\\
%&&$0.43<z<0.7$&$0.43<z<0.51$& $0.51<z<0.57$& $0.57<z<0.7$\\
%\hline
%1&0.05&$67.16\pm16.77$&$67.95\pm25.50$&$73.43\pm23.01$&$57.03\pm29.74$\\
%2&0.08&$58.98\pm6.73$&$54.83\pm10.58$&$63.67\pm13.04$&$59.08\pm14.40$\\
%\hline
%\end{tabular}
%\end{table*}

%%%%%%%%%%%%%%%%%%%%%%%%%%%%%%%%%%%%%%%%%%%%%%%%%%%%%%%%%%%%%%%%%%%%%%%%%%%%%%
%     Discussion
%%%%%%%%%%%%%%%%%%%%%%%%%%%%%%%%%%%%%%%%%%%%%%%%%%%%%%%%%%%%%%%%%%%%%%%%%%%%%%
\section{Discussion}\label{discussion}


%%%%%%%%%%%%%%%%%%%%%%%%%%%%%%%%%%%%%%%%%%%%%%%%%%%%%%%%%%%%%%%%%%%%%%%%%%%%%%
%     CONCLUSIONS
%%%%%%%%%%%%%%%%%%%%%%%%%%%%%%%%%%%%%%%%%%%%%%%%%%%%%%%%%%%%%%%%%%%%%%%%%%%%%%

\section{Summary and Conclusions}\label{conclusions}


\section*{Acknowledgements}


  
  



 %-------------- BIBLIO -------------------------------------------------------

\bibliographystyle{mnras}
\bibliography{mn-jour,all_refs}
\label{lastpage}

 %-----------------------------------------------------------------------------

\end{document}
